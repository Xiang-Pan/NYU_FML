\documentclass{article}
\usepackage[utf8]{inputenc}
\usepackage{bbm} 
\usepackage{tabularx}
\usepackage{amsopn}
\usepackage{amsmath}
\usepackage{amsfonts}
\usepackage{hyperref}
\usepackage{bm}
\title{%
   Homework 1 \\
  \large Fondations of Machine Learning \\}
\author{Xiang Pan (xp2030)}

\begin{document}


\maketitle

\section*{A Concentration of bound}
\section{}
\subsection{(a)}
According to the Hoeffding’s Inequality, we have

\begin{equation}
  \operatorname{Pr}[R(h)-\widehat{R}(h) \geq \epsilon] \leq e^{-2 m \epsilon^{2}}
\end{equation}

We use $\epsilon = \frac{1}{2}$ here, thus we have

\begin{equation}
  \operatorname{Pr}[R(h)-\widehat{R}(h) \geq \frac{1}{2}] \leq e^{-2 m \frac{1}{4}} = e^{-\frac{1}{2}m} < e^{-\frac{1}{3}m}
\end{equation}

So we will not have h,
\begin{equation}
  R(h)-\widehat{R}(h) \geq \frac{1}{2},
\end{equation}
with probability at least $e^{-\frac{1}{3}m}$.

\subsection{(b)}

The algorithm is that for all the samples in the training dataset, we assign label 1 to them, for any other samples, we assign label 0.

\begin{equation}
    R(h)=\operatorname{Pr}_{x \sim D}[h(x) \neq c(x)]=\underset{x \sim D}{\mathrm{E}}\left[1_{h(x) \neq c(x)}\right].
\end{equation}
  
We have $\widehat{R}_{S}\left(h_{S}\right)=0$, and ${R}(h_S)=1.$

According to the definition of $R(h)$, the training dataset samples is limited/finity, for the expectation, the final error is 1. So we have,

\begin{equation}
  R\left(h_{S}\right)-\widehat{R}_{S}\left(h_{S}\right)=1
\end{equation}

\subsection{(c)}

The (a) part is about the probability and the Hypothesis $h$ is not conditional on the data samples $S$. But for given Hypothesis and given data samples, we can design a algorithm to achieve the (b) part.

\section*{PAC-Bayesian bound}
\section{}
\subsection{(a)}
The Rademacher complexity bound is,
\begin{equation}
  \mathrm{E}[g(z)] \leq \frac{1}{m} \sum_{i=1}^{m} g\left(z_{i}\right)+2 \Re_{m}(G)+\sqrt{\frac{\log \frac{1}{\delta}}{2 m}}.
\end{equation}

We already know that $L(h,z) = l(h(x),y)$, L is a family of functions mapping from $ \mathbb{R} \times y \to [0,1]$. Directly apply the Rademacher complexity bound,

\begin{equation}
\underset{\underset{z \sim D}{h \sim Q}}{\mathbb{E}}[L(h, z)] \leq \underset{h \sim Q}{\mathbb{E}}\left[\frac{1}{m} \sum_{i=1}^{m} L\left(h, z_{i}\right)\right]+2 \Re_{m}\left(g_{\mu}\right)+\sqrt{\frac{\log \frac{1}{\delta}}{2 m}} \label{eq:2_1_a}
\end{equation}

\subsection{(b)}

Follow the result from \eqref{eq:2_1_a}, and the inequality from the statement,

\begin{equation}
  \mathfrak{R}_{m}\left(\mathcal{G}_{\mu}\right) \leq \sqrt{\frac{2 \mu}{m}}.
\end{equation}

We have, with probability at least $1-\delta$, and $Q \in \mathcal{G}_{\mu}$ 

\begin{equation}
  \underset{\underset{z \sim D}{h \sim Q}}{\mathbb{E}}[L(h, z)] \leq \underset{h \sim Q}{\mathbb{E}}\left[\frac{1}{m} \sum_{i=1}^{m} L\left(h, z_{i}\right)\right]+
  2 \sqrt{\frac{2 \mu}{m}}
  +\sqrt{\frac{\log \frac{1}{\delta}}{2 m}} \label{eq:r_comp}
\end{equation}



Comsidering some a,

\begin{equation}
  \Delta(\mathcal{H}) =  \{Q \in \Delta(\mathcal{H}): \mathrm{D}(Q \| P) \leq a\} 	\cup (\bigcup_{j=1}^{\infty}\left\{Q \in \Delta(\mathcal{H}): a 2^{j-1}<D(Q \| P) \leq a 2^{j}\right\})
\end{equation}

For a = 1,


\begin{equation}
    \Delta(\mathcal{H}) =  \{Q \in \Delta(\mathcal{H}): \mathrm{D}(Q \| P) \leq 1\} 	\cup (\bigcup_{j=1}^{\infty}\left\{Q \in \Delta(\mathcal{H}): 2^{j-1}<D(Q \| P) \leq 2^{j}\right\})
\end{equation}

We denote $\Delta{(H)} = \Delta{(H_0)} \cup \Delta{(H_1)} \cup ...\Delta{(H_{\infty})}   $,

where $\Delta(H_0) =\{Q \in \Delta(\mathcal{H}): \mathrm{D}(Q \| P) \leq 1\}$ 

and $\Delta(H_j)$ for $j>0$ denote ${Q \in \Delta(\mathcal{H}): 2^{j-1}<D(Q \| P) \leq 2^{j}}$.


The definition can be combined as,

\begin{equation}
    % \label{eq:alpha}
    \Delta(H_j) =
    \left\{
           \begin{array}{lcl}
            \{Q \in \Delta(\mathcal{H}): 0 \leq D(Q \| P) \leq 1\} & j=0\\
            \{Q \in \Delta(\mathcal{H}):  2^{j-1}<D(Q \| P) \leq 2^{j}\} &j \geq 1 
           \end{array}
        \right.
\end{equation}

% \begin{equation}
%     {\Delta(K_i)} =\{Q \in \Delta(\mathcal{H}): 0<D(Q \| P) \leq 2^{i}\}
% \end{equation}

Applying \eqref{eq:r_comp}, with probability at most $\delta_j = \frac{1}{(2^{j+1})}\delta$, and $Q \in \mathcal{G}_{2^j}$, we have,

\begin{align}
    \underset{\underset{z \sim D}{h \sim Q}}{\mathbb{E}}[L(h, z)] 
    % &> \underset{\underset{z \sim D}{h \sim Q, Q\in \Delta(K_i)}}{\mathbb{E}}[L(h, z)] \\
    &> \underset{h \sim Q}{\mathbb{E}}\left[\frac{1}{m} \sum_{i=1}^{m} L\left(h, z_{i}\right)\right]+
    2 \sqrt{\frac{2 (2^j)}{m}}
    +\sqrt{\frac{\log \frac{1}{\delta_j}}{2 m}} \label{eq:ori}
\end{align}

Note that $\Delta(H_j) \subseteq \mathcal{G}_{2^j}$, then for all  $Q \in \Delta(H_j)$, \eqref{eq:ori} still holds. And we name this inequality \eqref{eq:ori} as,

\begin{align}
    LHS > RHS_j.
\end{align}



\textbf{We want to prove (partial result, not final target):} with at least probability 1-$\delta_j$, we have,
\begin{align}
    \underset{\underset{z \sim D}{h \sim Q}}{\mathbb{E}}[L(h, z)] \leq \underset{h \sim Q}{\mathbb{E}}\left[\frac{1}{m} \sum_{i=1}^{m} L\left(h, z_{i}\right)\right]
    + ( 4 + \frac{1}{\sqrt{e}} )\sqrt{\frac{\max{(1,D(Q||P))}}{m}} + \sqrt{\frac{\log \frac{1}{\delta}}{2 m}}\label{eq:target}
\end{align}


\textbf{We can convert it to:} with at most probability $\delta_j$, we have,
\begin{align}
    \underset{\underset{z \sim D}{h \sim Q}}{\mathbb{E}}[L(h, z)] &> \underset{h \sim Q}{\mathbb{E}}\left[\frac{1}{m} \sum_{i=1}^{m} L\left(h, z_{i}\right)\right]
    + ( 4 + \frac{1}{\sqrt{e}} )\sqrt{\frac{\max{(1,D(Q||P))}}{m}} + \sqrt{\frac{\log \frac{1}{\delta}}{2 m}} \label{eq:LR_i}
\end{align}

We name it as
\begin{equation}
    LHS > RHS'.
\end{equation}

% for $\Delta(H_j)$.

We have with at most probability $\delta_j$, $LHS > RHS_j$ according to \eqref{eq:ori}, if we can prove that $RHS' > RHS_j$, thus we have with at most probability $\delta_j$, $LHS > RHS'$.

\textbf{We want to prove currently is $RHS' > RHS_j$,} when $Q \in \Delta(H_j)$.

\begin{align}
\begin{aligned}
    \iff
    & \underset{h \sim Q}{\mathbb{E}}\left[\frac{1}{m} \sum_{i=1}^{m} L\left(h, z_{i}\right)\right]
    + ( 4 + \frac{1}{\sqrt{e}} )\sqrt{\frac{\max{(1,D(Q||P))}}{m}} + \sqrt{\frac{\log \frac{1}{\delta}}{2 m}} \\
    &\geq \underset{h \sim Q}{\mathbb{E}}\left[\frac{1}{m} \sum_{i=1}^{m} L\left(h, z_{i}\right)\right]+
    2 \sqrt{\frac{2 \mu}{m}}
    +\sqrt{\frac{\log \frac{1}{\delta_j}}{2 m}}
\end{aligned}
\end{align}

\begin{align}
    \iff
    & 
    ( 4 + \frac{1}{\sqrt{e}} )\sqrt{\frac{\max{(1,D(Q||P))}}{m}} + \sqrt{\frac{\log \frac{1}{\delta}}{2 m}} 
    \geq 
    (2 \sqrt{\frac{2^{j+1}}{m}}
    +\sqrt{\frac{\log \frac{1}{\delta_j}}{2 m}})
\end{align}

\begin{align}
    \iff ( 4 + \frac{1}{\sqrt{e}} )\sqrt{\frac{\max{(1, 2^j) }}{m}} + \sqrt{\frac{\log \frac{1}{\delta}}{2 m}} 
    \geq 
    (2 \sqrt{\frac{2^{j+1}}{m}}
    +\sqrt{\frac{\log \frac{1}{\delta_j}}{2 m}}), \label{eq:20}
\end{align}



\begin{align}
    \iff ( 4 + \frac{1}{\sqrt{e}} )\sqrt{\frac{2^j }{m}} + \sqrt{\frac{\log \frac{1}{\delta}}{2 m}} 
    \geq 
    (2 \sqrt{\frac{{2^{j+1}}}{m}}
    +\sqrt{\frac{\log \frac{1}{\delta_j}}{2 m}})
\end{align}


\begin{align}
    \iff ( 4 + \frac{1}{\sqrt{e}} ) \sqrt{2^j}  + \sqrt{\frac{\log \frac{1}{\delta}}{2 }} 
    \geq 
    (2 \sqrt{2^{j+1}})
    +\sqrt{\frac{\log \frac{2^{j+1}}{\delta}}{2 }})
\end{align}


\begin{align}
    \iff ( 4 + \frac{1}{\sqrt{e}} ) \sqrt{2^j} + \sqrt{\frac{\log \frac{1}{\delta}}{2 }} 
    \geq 
    (2 \sqrt{2^{j+1}})
    +\sqrt{\frac{(j+1) + \log \frac{1}{\delta}}{2 }})
\end{align}

\begin{align}
    \iff ( \frac{1}{\sqrt{e}} ) \sqrt{2^j} + \sqrt{\frac{\log \frac{1}{\delta}}{2 }} 
    \geq 
    \sqrt{\frac{(j+1) + \log \frac{1}{\delta}}{2 }}) 
\end{align}

where \eqref{eq:20} is from $Q \in \Delta(H_j)$.

We have 
\begin{equation}
    \frac{t}{e} \geq \frac{\log (2 t)}{2}
\end{equation}

With $t = 2^i$, we have

\begin{align}
    \frac{2^j}{e} 
    &\geq \frac{\log (2^{j+1})}{2} \\
    &\geq \frac{(j+1)}{2}
\end{align}

\begin{align}
    % \iff 
    \sqrt{\frac{2^j}{e}}   + \sqrt{\frac{\log \frac{1}{\delta}}{2 }} 
    \geq 
    \sqrt{\frac{2^j}{e} + \frac{ \log \frac{1}{\delta}}{2 }}  
    \geq
    \sqrt{\frac{(j+1)}{2} + \frac{ \log \frac{1}{\delta}}{2 }} \label{eq:final}
\end{align}


We have proved \eqref{eq:final}, thus we have proved the \eqref{eq:LR_i}.

We name the inequality \eqref{eq:target} holds as event $E_j = 1$, and the inequality does not hold as $E_j = 0$, the final inequality \eqref{eq:final_target} holds as event $E=1$, the final inequality does not hold as $E=0$.

\begin{align}
&\operatorname{Pr}\left[E=0\right] \\
&=\operatorname{Pr}\left[\bigcup_{j \in [0,\infty]} Q \in \Delta(H_j): E_j=0 \right] \\
&\leq\sum_{j \in [0,\infty]} \operatorname{Pr}\left[ Q \in \Delta(H_j) :E_j=0 \right] && \text{(Union Bound)}\\
&=\sum_{j \in [0,\infty]} \operatorname{Pr}\left[ Q \in \Delta(H_j): LHS > RHS' \right] \\
&\leq\sum_{j \in [0,\infty]} \operatorname{Pr}\left[ Q \in \Delta(H_j) : LHS > RHS_j \right] && (RHS' > RHS_i) \\ 
% &\leq \sum_{h \in H_{\epsilon}} \operatorname{Pr}\left[\widehat{R}_{S}(h)=0\right] \\
&= \delta_0 + \delta_1 + ... \delta_{\infty} && \text{(Probability from \eqref{eq:ori})} \\
&= (1/2 + 1/4 + 1/8 + ... + 1/2^{\infty}) \delta  \\
&= (\sum_{j=0}^{\infty} \frac{1}{2^j})\delta \\
&= \delta
\end{align}

With at most probability $\delta$, we have proved,.
\begin{equation}
    \underset{h \sim Q \atop z \sim \mathcal{D}}{\mathbb{E}}[L(h, z)] > \underset{h \sim Q}{\mathbb{E}}\left[\frac{1}{m} \sum_{i=1}^{m} L\left(h, z_{i}\right)\right]+\left(4+\frac{1}{\sqrt{e}}\right) \sqrt{\frac{\max \{\mathrm{D}(Q \| P), 1\}}{m}}+\sqrt{\frac{\log \frac{1}{\delta}}{2 m}}
\end{equation}

With at least probability 1-$\delta$, we have proved,
\begin{equation}
    \underset{h \sim Q \atop z \sim D}{\mathbb{E}}[L(h, z)] \leq \underset{h \sim Q}{\mathbb{E}}\left[\frac{1}{m} \sum_{i=1}^{m} L\left(h, z_{i}\right)\right]+\left(4+\frac{1}{\sqrt{e}}\right) \sqrt{\frac{\max \{\mathrm{D}(Q \| P), 1\}}{m}}+\sqrt{\frac{\log \frac{1}{\delta}}{2 m}}\label{eq:final_target}
\end{equation}

% Where \eqref{eq:qqq} is from (RHS' > RHS_i).

\newpage
\section*{Rademacher complexity}
\section{}
\subsection{(a)}

\begin{align}
    \widehat{\Re}_{\mathcal{S}}(\mathcal{H})&=\underset{\boldsymbol{\sigma}}{\mathbb{E}}\left[\sup _{h \in \mathcal{H}} \frac{1}{m} \sum_{i=1}^{m} \sigma_{i} h\left(\mathbf{x}_{i}\right)\right] \\
    &=\frac{1}{m}
    \underset{\boldsymbol{\sigma}}{\mathbb{E}}
    \left[
        \sup _{h \in \mathcal{H}}  \mathbf{w} \sum_{i=1}^{m} \sigma_{i}  \mathbf{x_i}
    \right] \\
    % &= \frac{\Lambda}{m} \underset{\boldsymbol{\sigma}}{\mathbb{E}}
    % \left[
    %     \sup _{h \in \mathcal{H}} \sum_{i=1}^{m} \sigma_{i} x_i
    % \right] \\
    &= \frac{\Lambda}{m} \underset{\boldsymbol{\sigma}}{\mathbb{E}}
    \left[
    \left\|\sum_{i=1}^{m} \sigma_{i} \mathbf{x}_{i}\right\|_{\infty}
    \right] && \text{(by the definition of dual norm)} \\
    &= \frac{\Lambda}{m} \underset{\boldsymbol{\sigma}}{\mathbb{E}}
    \left[
        \underset{j \in d}{\max}
        \left|
            \sum_{i=1}^{m} \sigma_{i} {x}_{ij}
        \right|
    \right]  \\
    &= \frac{\Lambda}{m} \underset{\boldsymbol{\sigma}}{\mathbb{E}}
    \left[
        \underset{j \in d}{\max}
        \underset{s \in \{-1, +1\}}{\max}
            s \sum_{i=1}^{m} \sigma_{i} {x}_{ij}
    \right]  \\
    &= \frac{\Lambda}{m} \underset{\boldsymbol{\sigma}}{\mathbb{E}}
    \left[
        \underset{\mathbf{z} \in A}{\sup}
            \sum_{i=1}^{m} \sigma_{i} z_i
    \right]  
    % &= \frac{\Lambda}{m} \underset{\boldsymbol{\sigma}}{\mathbb{E}}
    % \left[
    %     \underset{\mathbf{z} \in A}{\sup}
    %         \sum_{i=1}^{m} \sigma_{i} z_i
    % \right]  
\end{align}

Where $A$ is the column vector set in M.

\begin{equation}
    A := \left\{s\left(x_{1 j}, \ldots, x_{m j}\right)^{\top}: j \in[d], s \in\{-1,+1\}\right\}
\end{equation}

For any $z \in A$, we have,
\begin{equation}
    \|\mathbf{z}\|_{2} \leq \sup _{\mathbf{z} \in A}\|\mathbf{z}\|_{2}=\left\|\mathbf{X}^{\top}\right\|_{2, \infty}
\end{equation}

Using Massart's lemma, A contains at most 2N elements, we have,
\begin{equation}
    \widehat{\Re}_{S}(\mathcal{H}) \leq \frac{\Lambda}{m} \sqrt{2 \log (2 N)}\left\|X^{\top}\right\|_{2, \infty} .
\end{equation}

\newpage

\subsection{(b)}

% \begin{align}
%   \widehat{\Re}_{\mathcal{S}}\left(\mathcal{F}_{p}\right) &=\frac{1}{m} \underset{\sigma}{\mathbb{E}}\left[\sup _{\|\mathbf{w}\|_{p} \leq \Lambda} \mathbf{w} \cdot \sum_{i=1}^{m} \sigma_{i} \mathbf{x}_{i}\right] \\
%   &=\frac{\Lambda}{m} \underset{\boldsymbol{\sigma}}{\mathbb{E}}\left[\left\|\mathbf{u}_{\boldsymbol{\sigma}}\right\|_{p^{*}}\right] \\
%   & \leq \frac{\Lambda}{m}\left[\underset{\boldsymbol{\sigma}}{\mathbb{E}}\left[\left\|\mathbf{u}_{\boldsymbol{\sigma}}\right\|_{p^{*}}^{p^{*}}\right]\right]^{\frac{1}{p^{*}}} && \text{(Jensen’s inequality)} \\
%   &=\frac{\Lambda}{m}\left[\sum_{j=1}^{d} \underset{\boldsymbol{\sigma}}{\mathbb{E}}\left[\left|\mathbf{u}_{\boldsymbol{\sigma}, j}\right|^{p^{*}}\right]\right]^{\frac{1}{p^{*}}} 
% \end{align}

\textbf{Case 1: $p \leq 2$}
By directly applying Jensen's inequality, we get,

$C_1 = C_2 = 1,$

\begin{equation}
    \underset{\mathbf{\sigma}}{\mathbb{E}}\left[\left|\sum_{i=1}^{m} \sigma_{i} a_{i}\right|^{p}\right] \leq \left(\sum_{i=1}^{m} a_{i}^{2}\right)^{\frac{p}{2}}.
\end{equation}


\textbf{Case 2: $p > 2$}

\begin{align}
    \mathbb{E}\left[e^{\sigma \mathbf{a}}\right]
    &=\prod_{i=1}^{m} \mathbb{E}
        \left[
            e^{\sigma_i a_i}
        \right]\\
    &=\prod_{i=1}^{m} 
        \frac{e^{a_i}+e^{-a_i}}{2}\\
    &=\prod_{i=1}^{m} 
        \cosh 
            \left(
             a_i
            \right) \\
    &\leq
    e^{\frac{1}{2} \sum_{i=1}^{m}  a_{i}^{2}} && (\cosh (x) \leq e^{\frac{1}{2} x^{2}})\\ 
    &\leq
    e^{ \frac{1}{2} \|\bm{a}\|_{2}^{2} }
\end{align}

\begin{equation}
    \begin{aligned}
        \mathbb{E}[\cosh (\bm{a} \cdot \bm{\sigma})] 
        &=\frac{1}{2} \mathbb{E}\left[e^{\bm{a} \cdot \bm{\sigma}}+e^{-\bm{a} \cdot \bm{\sigma}}\right] \\
        & \leq e^{\frac{1}{2} \|\bm{a}\|_{2}^{2}}
    \end{aligned}
\end{equation}

Since $\cosh (x)$ grows faster than $x^{p}$, for any $0<p<\infty$, there exists a positive constant $B_{p}$ satisfying $|x|^{p} \leq B_{p} \cosh (x)$. Hence,

\begin{equation}
    {\mathbb{E}\left[| \mathbf{a} \cdot \bm{\sigma} |^{p}\right]} \leq B_{p} \mathbb{E}[\cosh (\mathbf{a})] \leq B_{p} e^{\frac{1}{2} \|\mathbf{a}\|_{2}^{2}}
\end{equation}

\begin{equation}
    \mathbb{E}\left[|\mathbf{a} \cdot \bm{\sigma}|^{p}\right] \leq C_{p}\|\mathbf{a}\|_{2}^{p}
\end{equation}

\begin{equation}
    \underset{\mathbf{\sigma}}{\mathbb{E}}\left[\left|\sum_{i=1}^{m} \sigma_{i} a_{i}\right|^{p}\right] \leq C_P \left(\sum_{i=1}^{m} a_{i}^{2}\right)^{\frac{p}{2}}.
\end{equation}

\begin{equation}
    C_{p}= B_{p} \exp \left( 1/2\right)
\end{equation}

\newpage

\subsection{(c)}

\begin{equation}
    c_{p}\left(\sum_{i=1}^{m} a_{i}^{2}\right)^{\frac{p}{2}} \leq \underset{\sigma}{\mathbb{E}}\left[\left|\sum_{i=1}^{m} \sigma_{i} a_{i}\right|^{p}\right]
\end{equation}

\textbf{Case 1: $p \geq 2$}

Directly applying Jensen's inequality, we get,

\begin{equation}
    \mathbb{E}\left[|a \cdot \sigma|^{p}\right] \geq \mathbb{E}\left[(a \cdot \sigma)^{2}\right]^{p / 2}=\|a\|_{2}^{p}
\end{equation}

$c_p = 1 \text{ for } p\geq2$

\textbf{Case 2: $p < 2$}

We choose $q > 2$,

\begin{equation}
    \begin{aligned}
    \|a\|_{2}^{2(q-p)} &=\mathbb{E}\left[|a \cdot \sigma|^{2}\right]^{q-p} \\
    & \leq \mathbb{E}\left[|a \cdot \sigma|^{p}\right]^{q-2} \mathbb{E}\left[|a \cdot \sigma|^{q}\right]^{2-p} \\
    & \leq \mathbb{E}\left[|a \cdot \sigma|^{p}\right]^{q-2} C_{q}^{2-p}\|a\|_{2}^{q(2-p)}
    \end{aligned}
\end{equation}

\begin{align}
    \mathbb{E}\left[|a \cdot \sigma|^{p}\right] \geq C_{q}^{(p-2)/(q-2)} \|a\|_{2}^{p},
\end{align}

By using the $\|a\|_2 = 1$.

\begin{equation}
    c_{p}=C_{q}^{(p-2) /(q-2)},
\end{equation}

where $C_{q}$ is the constant in part (b).


\subsection{(d)}

% For $p=1$, 

\begin{equation}
    \begin{aligned}
    \widehat{\mathcal{R}}_{\mathcal{S}}
    \left(H\right) &=\underset{\boldsymbol{\sigma}}{\mathbb{E}}
    \left[\sup _{\|\mathbf{w}\|_{1} \leq \Lambda} \mathbf{w} \cdot \sum_{i=1}^{m} \sigma_{i} \mathbf{x}_{i}\right] \\
    &=\frac{\Lambda}{m} 
    \underset{\boldsymbol{\sigma}}{\mathbb{E}}
        \left[
            \sup _{\|\mathbf{w}\|_{1} \leq \Lambda}
            \left| \sum_{i=1}^{m} \sigma_{i} \mathbf{x}_{i} \right| 
        \right]\\
    &\geq c_1 \frac{\Lambda}{m} 
    \left[
        \sup _{\|\mathbf{w}\|_{1} \leq \Lambda} \|x\|_{2}^{p} 
    \right] \\
    &= c_1 \frac{\Lambda}{m}\left\|X^{\top}\right\|_{2, \infty} \\
    % &= c_1 \frac{\Lambda}{m}   
    % & \geq \frac{\Lambda}{m}\left\|\mathbb{E}{\mathbb{\sigma}}\left[\left|\sum_{i=1}^{m} \sigma_{i} \mathbf{x}_{i}\right|\right]\right\|_{p^{*}} \\
    % &=\frac{\Lambda}{m}\left[\sum_{j=1}^{d}\left(\underset{\boldsymbol{\sigma}}{\mathbb{E}}\left[\left|\sum_{i=1}^{m} \sigma_{i} \mathbf{x}_{i j}\right|\right]\right)^{p^{*}}\right]^{\frac{1}{p^{*}}}
    \end{aligned}
\end{equation}

\begin{equation}
    \widehat{\Re}_{S}(\mathcal{H}) \geq c_{1} \frac{\Lambda}{m}\left\|X^{\top}\right\|_{2, \infty}
\end{equation}




\subsection{(e)}

We consider the dataset with dimension $N = 2^m$ and all the elements of data $x_{ij} \in \{-1,1\}$, so the matrix $X \in \mathbf\{-1,1\}^{2^m \times m}$.

\begin{align}
\widehat{\mathfrak{R}}_{\mathcal{S}}\left(H\right)
&=\frac{1}{m} \underset{\boldsymbol{\sigma}}{\mathbb{E}}\left[\sup _{\|\mathbf{w}\|_{1} \leq \Lambda} \mathbf{w} \cdot \sum_{i=1}^{m} \sigma_{i} \mathbf{x}_{i}\right] \\
&=\frac{\Lambda}{m} \underset{\boldsymbol{\sigma}}{\mathbb{E}}\left[\left\|\sum_{i=1}^{m} \sigma_{i} \mathbf{x}_{i}\right\|_{\infty}\right] \\
% &=\frac{1}{m} \underset{\boldsymbol{\sigma}}{\mathbb{E}}
% \begin{equation}
&=\frac{\Lambda}{m} \underset{\boldsymbol{\sigma}}{\mathbb{E}}\left[\max _{j \in [1,d]} \sum_{i=1}^{m} \sigma_{i}\left(\mathbf{x}_{i}\right)_{j}\right] \\
&=\frac{\Lambda}{m} m \\
&=\Lambda \\
&=\frac{\Lambda}{m} \sqrt{m} \sqrt{m} \\
&=\frac{\Lambda}{m} \sqrt{\log{N}} {\|X^T|\|_{2, \infty}}
\end{align} 

\begin{align}
    {\|X^T|\|_{2, \infty}} 
    &= \left\|\left(\left\|{X^T}_{1}\right\|_{2}, \ldots,\left\|{X^T}_{N}\right\|_{2}\right)\right\|_{\infty} \\
    &= \sqrt{m}
\end{align}

We have given a example and showed the upper bound is tight for $p=1$.


\section{Note}

I notice something strange to me:

In paper \cite{awasthi2020rademacher}, the (p,q) group norm is given as

\begin{equation}
    \|\mathbf{M}\|_{p, q}=\left\|\left(\left\|\mathbf{M}_{1}\right\|_{1}, \ldots,\left\|\mathbf{M}_{d}\right\|_{p}\right)\right\|_{q}
\end{equation}

which is different from the (p,q) group norm definition here.


\bibliographystyle{plain}
\bibliography{main}
\nocite*{}




\end{document}
